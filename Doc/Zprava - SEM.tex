\documentclass[10pt,a4paper]{article}
\usepackage[latin1]{inputenc}
\usepackage[english]{babel}
\usepackage{amsmath}
\usepackage{amsfonts}
\usepackage{amssymb}
\usepackage{graphicx}
\usepackage[usenames,dvipsnames,svgnames,table]{xcolor}
\usepackage{fullpage}
\usepackage{url}
\usepackage{hyperref}
\usepackage{listings}
\title{SEM model - první výsledky}

\lstset{
language=R,
basicstyle=\scriptsize\ttfamily,
commentstyle=\ttfamily\color{gray},
numbers=none,
numberstyle=\ttfamily\color{gray}\footnotesize,
stepnumber=1,
numbersep=5pt,
backgroundcolor=\color{white},
showspaces=false,
showstringspaces=false,
showtabs=false,
frame=single,
tabsize=2,
captionpos=b,
breaklines=true,
breakatwhitespace=false,
title=\lstname,
escapeinside={},
keywordstyle={},
morekeywords={}
}

\begin{document}
\section{Descriptive statistics}
Before any more sophisticated analysis descriptive statistics should be computed. It allows for identification of groups (social, economical, sex, occupation) with different attitudes. As these are not of our concern, be should make clear they do not differ. If they do, we should create separate models for these groups or exclude different group from analysis.

From what I inspected (but not documented here) it's not obvious whether any group has different response pattern. In this latent construct (loyalty) we have two contradicting answers on question two and three (it seems respondents want both switch to another bank and stay with the current).

\begin{figure}[htbp]
\includegraphics[trim=0.8cm 8cm 0.25cm 8cm, clip, width=1.00\textwidth]{Rplot1}
\caption{Plot is divided into positive and negative (left side) answers by gray zone of neutral answer. Proportion of Negative answers in the first question was 27\% whereas positive rate reached 43\%.}
\end{figure}

\begin{figure}
\includegraphics[trim=0.8cm 8cm 0.25cm 8cm, clip, width=0.9\textwidth]{Rplot2}
\caption{This plot conveys the same information but in different layout. This is not preferable in psychometrics as it does not allow for direct comparison negative and positive answers. However, it is usually reported in papers \dots }
\end{figure}
\begin{lstlisting}
# Read data from "sem_prepare_data.R"
library(likert)
library(reshape) #without reshape errors accur in grouping
plot.age <- likert(data.sem[ ,25:28], grouping=data.in$age)
plot(plot.edu) # first "nice" plot
plot(plot.edu, centered=FALSE) # second plot
\end{lstlisting}

It might look suspiciously that responses of our questionnaire do not contain any missing values.

\subsection{Exploratory Factor Analysis}
Factor analysis is used to determine number of constructs in the data based on the shared variance of manifest variables (measured variables, our questions). We can either set in advance number of factors and let the program to classify questions to estimated factors or let the program identify number of factors by some stepping rule. This step is crucial as it supports discriminant validity of construct (the difference between constructs is clear). As a perfect results would be a matrix with high loadings of questions related to the particular factor (estimated from data). Manifest variables of one group (i.e., akce) should be grouped within one factor.

\subsubsection{Factor Analysis with k=8 factors}
We start with our hypothesised model of 8 factors. We perform series of test related to this mode:
\begin{lstlisting}
fac.an1 <- fa(cor.mat, 8) 
om1 <- omega(cor.mat, 8) # TODO
fa.sort(fac.an1) # print ordered results of loadings
\end{lstlisting}

% Table generated by Excel2LaTeX from sheet 'List2'

%\textcolor{Gray}{0.1} 
% \textcolor{Red}{
% \textcolor{Orange}{
% \textcolor{NavyBlue}{0.52}

\begin{table}[htbp]
  \centering
  \caption{Standardized loadings (pattern matrix) based upon correlation matrix. Loadings which exceed~0.4 can be considered as member of factor group. If loadings of manifest variable exceed 0.3 then the question can be considered as a members of factor but only if properly described in text(i.e., with reference to other sources \dots}
    \begin{tabular}{r|rrrrrrrr|rrr}
        & F3 & F1 & F2 & F4 & F6 & F8 & F7 & F5 & h2  & u2  & com \\\hline
    loaj2 & \textcolor{NavyBlue}{0.85} & \textcolor{Gray}{-0.03} & \textcolor{Gray}{-0.03} & \textcolor{Gray}{0.05} & \textcolor{Gray}{0   } & \textcolor{Gray}{0.03} & \textcolor{Gray}{-0.01} & \textcolor{Gray}{-0.01} & 0.73 & 0.27 & 1 \\
    loaj3 & \textcolor{NavyBlue}{0.83} & \textcolor{Gray}{0.02} & \textcolor{Gray}{-0.03} & \textcolor{Gray}{0.01} & \textcolor{Gray}{-0.03} & \textcolor{Gray}{0.04} & \textcolor{Gray}{0.06} & \textcolor{Gray}{0.02} & 0.73 & 0.27 & 1 \\
    spok4 & \textcolor{Orange}{0.46} & \textcolor{Gray}{0.11} & \textcolor{Gray}{0.29} & \textcolor{Gray}{-0.02} & \textcolor{Gray}{0.18} & \textcolor{Gray}{-0.17} & \textcolor{Gray}{0   } & \textcolor{Gray}{-0.01} & 0.54 & 0.46 & 2.5 \\
    spok3 & \textcolor{Orange}{0.42} & \textcolor{Gray}{0.16} & \textcolor{Gray}{0.14} & \textcolor{Gray}{0.16} & \textcolor{Gray}{0.07} & \textcolor{Gray}{-0.1} & \textcolor{Gray}{0.05} & \textcolor{Gray}{0.03} & 0.55 & 0.45 & 2.1 \\
    loaj4 & \textcolor{NavyBlue}{0.38} & \textcolor{Gray}{0.13} & \textcolor{Gray}{-0.02} & \textcolor{Gray}{-0.01} & \textcolor{Gray}{-0.05} & \textcolor{Gray}{0.11} & \textcolor{Gray}{0.29} & \textcolor{Gray}{0   } & 0.42 & 0.58 & 2.4 \\
    akce2 & \textcolor{Gray}{0.04} & \textcolor{NavyBlue}{0.67} & \textcolor{Gray}{-0.08} & \textcolor{Gray}{0.19} & \textcolor{Gray}{-0.02} & \textcolor{Gray}{-0.01} & \textcolor{Gray}{-0.06} & \textcolor{Gray}{0.02} & 0.59 & 0.41 & 1.2 \\
    akce1 & \textcolor{Gray}{0.05} & \textcolor{NavyBlue}{0.65} & \textcolor{Gray}{0   } & \textcolor{Gray}{0.08} & \textcolor{Gray}{0.07} & \textcolor{Gray}{0.03} & \textcolor{Gray}{0.05} & \textcolor{Gray}{-0.16} & 0.63 & 0.37 & 1.2 \\
    akce4 & \textcolor{Gray}{0.01} & \textcolor{NavyBlue}{0.6} & \textcolor{Gray}{0.07} & \textcolor{Gray}{-0.04} & \textcolor{Gray}{0.02} & \textcolor{Gray}{-0.06} & \textcolor{Gray}{0.18} & \textcolor{Gray}{0.14} & 0.48 & 0.52 & 1.4 \\
    akce3 & \textcolor{Gray}{0.08} & \textcolor{NavyBlue}{0.51} & \textcolor{Gray}{-0.05} & \textcolor{Gray}{0.06} & \textcolor{Gray}{0.13} & \textcolor{Gray}{0.08} & \textcolor{Gray}{-0.1} & \textcolor{Gray}{0.1 } & 0.4 & 0.6 & 1.5 \\
    spok1 & \textcolor{Gray}{0.21} & \textcolor{Orange}{0.4}  & \textcolor{Gray}{0.21} & \textcolor{Gray}{-0.1} & \textcolor{Gray}{0.06} & \textcolor{Gray}{-0.06} & \textcolor{Gray}{0.12} & \textcolor{Gray}{0.02} & 0.43 & 0.57 & 2.7 \\
    fipo4 & \textcolor{Gray}{0.01} & \textcolor{Orange}{0.31} & \textcolor{Gray}{0.29} & \textcolor{Gray}{0.13} & \textcolor{Gray}{0.01} & \textcolor{Gray}{0.1 } & \textcolor{Gray}{0.06} & \textcolor{Gray}{-0.23} & 0.45 & 0.55 & 3.6 \\
    inpr2 & \textcolor{Gray}{0.06} & \textcolor{Gray}{-0.12} & \textcolor{NavyBlue}{0.66} & \textcolor{Gray}{0.03} & \textcolor{Gray}{0.02} & \textcolor{Gray}{0.04} & \textcolor{Gray}{0.11} & \textcolor{Gray}{0.06} & 0.54 & 0.46 & 1.2 \\
    fipo1 & \textcolor{Gray}{-0.02} & \textcolor{Gray}{0.19} & \textcolor{Orange}{0.56} & \textcolor{Gray}{0.04} & \textcolor{Gray}{0.02} & \textcolor{Gray}{0.18} & \textcolor{Gray}{0   } & \textcolor{Gray}{0.01} & 0.55 & 0.45 & 1.5 \\
    inpr3 & \textcolor{Gray}{0.06} & \textcolor{Gray}{0.01} & \textcolor{NavyBlue}{0.51} & \textcolor{Gray}{0.08} & \textcolor{Gray}{-0.02} & \textcolor{Gray}{0.32} & \textcolor{Gray}{0   } & \textcolor{Gray}{-0.13} & 0.57 & 0.43 & 1.9 \\
    kval4 & \textcolor{Gray}{0.13} & \textcolor{Gray}{-0.09} & \textcolor{Brown}{0.38} & \textcolor{Gray}{0.21} & \textcolor{Gray}{0.26} & \textcolor{Gray}{-0.05} & \textcolor{Gray}{-0.01} & \textcolor{Gray}{0.09} & 0.45 & 0.55 & 3 \\
    fipo3 & \textcolor{Gray}{-0.05} & \textcolor{Gray}{0.21} & \textcolor{Orange}{0.36} & \textcolor{Gray}{0.13} & \textcolor{Gray}{0   } & \textcolor{Gray}{0.1 } & \textcolor{Gray}{0.08} & \textcolor{Gray}{0.21} & 0.42 & 0.58 & 3 \\
    loaj1 & \textcolor{Gray}{0.05} & \textcolor{Gray}{0.02} & \textcolor{Gray}{-0.01} & \textcolor{NavyBlue}{0.86} & \textcolor{Gray}{0.02} & \textcolor{Gray}{0.01} & \textcolor{Gray}{0.02} & \textcolor{Gray}{-0.03} & 0.83 & 0.17 & 1\\
    spok2 & \textcolor{Gray}{0   } & \textcolor{Gray}{0.14} & \textcolor{Gray}{0.06} & \textcolor{Orange}{0.52} & \textcolor{Gray}{-0.04} & \textcolor{Gray}{-0.07} & \textcolor{Gray}{0.04} & \textcolor{Gray}{0.12} & 0.41 & 0.59 & 1.4 \\
    duve3 & \textcolor{Gray}{-0.05} & \textcolor{Gray}{0.01} & \textcolor{Gray}{-0.07} & \textcolor{Gray}{0.03} & \textcolor{NavyBlue}{0.61} & \textcolor{Gray}{0.01} & \textcolor{Gray}{0.16} & \textcolor{Gray}{0.07} & 0.43 & 0.57 & 1.2 \\
    kval3 & \textcolor{Gray}{0.14} & \textcolor{Gray}{-0.01} & \textcolor{Gray}{0.1 } & \textcolor{Gray}{-0.02} & \textcolor{Orange}{0.45} & \textcolor{Gray}{0.04} & \textcolor{Gray}{-0.01} & \textcolor{Gray}{-0.18} & 0.32 & 0.68 & 1.7 \\
    duve1 & \textcolor{Gray}{-0.03} & \textcolor{Gray}{0.18} & \textcolor{Gray}{-0.09} & \textcolor{Gray}{0.03} & \textcolor{NavyBlue}{0.37} & \textcolor{Gray}{0.02} & \textcolor{Gray}{0.04} & \textcolor{Gray}{-0.28} & 0.29 & 0.71 & 2.6 \\
    duve2 & \textcolor{Gray}{0.08} & \textcolor{Gray}{0.02} & \textcolor{Gray}{-0.07} & \textcolor{Gray}{0.1 } & \textcolor{NavyBlue}{0.34} & \textcolor{Gray}{0.19} & \textcolor{Gray}{-0.05} & \textcolor{Gray}{0.22} & 0.28 & 0.72 & 2.9 \\
    kval1 & \textcolor{Gray}{0   } & \textcolor{Gray}{0.15} & \textcolor{Gray}{0.23} & \textcolor{Gray}{0.21} & \textcolor{Orange}{0.32} & \textcolor{Gray}{0.07} & \textcolor{Gray}{0.06} & \textcolor{Gray}{0.04} & 0.51 & 0.49 & 3.4 \\
    kval2 & \textcolor{Gray}{0.12} & \textcolor{Gray}{0.21} & \textcolor{Gray}{0.07} & \textcolor{Gray}{0.06} & \textcolor{Orange}{0.25} & \textcolor{Gray}{0.11} & \textcolor{Gray}{-0.04} & \textcolor{Gray}{-0.19} & 0.34 & 0.66 & 4.2 \\
    duve4 & \textcolor{Gray}{0.12} & \textcolor{Gray}{0.08} & \textcolor{Gray}{0.19} & \textcolor{Gray}{0.03} & \textcolor{NavyBlue}{0.25} & \textcolor{Gray}{0.19} & \textcolor{Gray}{-0.01} & \textcolor{Gray}{0.23} & 0.4 & 0.6 & 4.5 \\
    inpr4 & \textcolor{Gray}{0.04} & \textcolor{Gray}{0.02} & \textcolor{Gray}{-0.06} & \textcolor{Gray}{0.04} & \textcolor{Gray}{0.07} & \textcolor{NavyBlue}{0.62} & \textcolor{Gray}{0.07} & \textcolor{Gray}{0.07} & 0.44 & 0.56 & 1.1 \\
    inpr1 & \textcolor{Gray}{0.04} & \textcolor{Gray}{-0.04} & \textcolor{Gray}{0.18} & \textcolor{Gray}{-0.07} & \textcolor{Gray}{-0.02} & \textcolor{NavyBlue}{0.6} & \textcolor{Gray}{0.02} & \textcolor{Gray}{-0.05} & 0.46 & 0.54 & 1.2 \\
    crse3 & \textcolor{Gray}{0.05} & \textcolor{Gray}{-0.01} & \textcolor{Gray}{-0.02} & \textcolor{Gray}{0.01} & \textcolor{Gray}{0.06} & \textcolor{Gray}{0.03} & \textcolor{NavyBlue}{0.71} & \textcolor{Gray}{0   } & 0.58 & 0.42 & 1 \\
    crse4 & \textcolor{Gray}{0.1 } & \textcolor{Gray}{0.05} & \textcolor{Gray}{0.04} & \textcolor{Gray}{0.19} & \textcolor{Gray}{0.12} & \textcolor{Gray}{0.05} & \textcolor{NavyBlue}{0.34} & \textcolor{Gray}{-0.23} & 0.45 & 0.55 & 3.1 \\
    crse1 & \textcolor{Gray}{0.03} & \textcolor{Gray}{-0.05} & \textcolor{Gray}{0.08} & \textcolor{Gray}{0.32} & \textcolor{Gray}{-0.02} & \textcolor{Gray}{0.1 } & \textcolor{NavyBlue}{0.33} & \textcolor{Gray}{0.07} & 0.37 & 0.63 & 2.5 \\
    crse2 & \textcolor{Gray}{0.13} & \textcolor{Gray}{0.12} & \textcolor{Gray}{0.09} & \textcolor{Gray}{0.19} & \textcolor{Gray}{-0.1} & \textcolor{Gray}{-0.12} & \textcolor{NavyBlue}{0.24} & \textcolor{Gray}{-0.03} & 0.25 & 0.75 & 4.6 \\
    fipo2 & \textcolor{Gray}{0.1 } & \textcolor{Gray}{0.12} & \textcolor{Red}{0.32} & \textcolor{Gray}{-0.01} & \textcolor{Gray}{0.15} & \textcolor{Gray}{0.06} & \textcolor{Gray}{0.04} & \textcolor{NavyBlue}{0.34} & 0.46 & 0.54 & 3\\
    \end{tabular}
  \label{t:load1}%
\end{table}%

From the Table \ref{t:load1} we can read no clear distinctive patterns between manifest variables. Factor \texttt{F3} might be named \emph{loaj} but \emph{spok} are also present in this factor. It seems clear that \texttt{F1} is \emph{akce} but there is one 0.4 loading of \emph{spok} and 0.31 \emph{fipo}.  These are the two most important factors, as shown in \ref{t:load1}.

% Table generated by Excel2LaTeX from sheet 'List2'
\begin{table}[htbp]
  \centering
  \caption{Table depicts the importance of factors in terms of explained variance. SS loadings represents sum of loadings squares, Proportion Explained is should be larger than $1/8 = 0.125$ to bring more information to our analysis than if it was randomly chosen.}
    \begin{tabular}{rrrrrrrrr}
          & MR3   & MR1   & MR2   & MR4   & MR6   & MR8   & MR7   & MR5 \\
    SS loadings & 2.83  & 2.76  & 2.48  & 2.07  & 1.72  & 1.34  & 1.45  & 0.66 \\
    Proportion Var & 0.09  & 0.09  & 0.08  & 0.06  & 0.05  & 0.04  & 0.05  & 0.02 \\
    Cumulative Var & 0.09  & 0.17  & 0.25  & 0.32  & 0.37  & 0.41  & 0.46  & 0.48 \\
    \textcolor{NavyBlue}{Proportion Explained} & 0.18  & 0.18  & 0.16  & 0.13  & 0.11  & 0.09  & 0.09  & 0.04 \\
    Cumulative Proportion & 0.18  & 0.36  & 0.53  & 0.66  & 0.77  & 0.86  & 0.96  & 1 \\
    \end{tabular}%
  \label{t:var1}%
\end{table}%

\subsubsection{Factor Analysis with \emph{optimal} factors}
In this section we identify ideal number of factors in data and we will also describe factor loadings.
\begin{lstlisting}
library(rela)
pca.ident <- paf(object=data.mat, eigcrit=1) # principal axis factor analysis
\end{lstlisting}
There are several approaches to setting appropriate number of factors. In this sub-section we present results of 5 factor model. We set $k=5$ from \emph{pca.ident} results (based on eigenvalue criteria). This is $\sim$ supported by Table \ref{t:var1} and Proportion Explained of 0.11 (which is only slightly below 0.125) of the fifth factor. 

\begin{lstlisting}
library(rela)
pca.ident <- paf(object=data.mat, eigcrit=1) # principal axis factor analysis
\end{lstlisting}

% Table generated by Excel2LaTeX from sheet 'List3'
\begin{table}[htbp]
  \centering
  \caption{Factor loadings matrix. As the number of factor decreased some questions on different groups had to necessarily merge. As a result we observe that the most influential factor \texttt{F1} now consists of loyalty, satisfaction and all cross-selling questions. Theses results indicates that respondents do not distinguish enough between these concepts. This is also a supportive argument for low level of discriminant validity.}
    \begin{tabular}{r|rrrrrrrr}
          & F1   & F3   & F2   & F5   & F4   & h2    & u2    & com \\\hline
    loaj3 & \textcolor{NavyBlue}{0.87} & \textcolor{Gray}{-0.02} & \textcolor{Gray}{-0.01} & \textcolor{Gray}{-0.02} & \textcolor{Gray}{0} & 0.71  & 0.29  & 1 \\
    loaj2 & \textcolor{NavyBlue}{0.87} & \textcolor{Gray}{-0.04} & \textcolor{Gray}{-0.02} & \textcolor{Gray}{0.01} & \textcolor{Gray}{-0.03} & 0.7   & 0.3   & 1 \\
    loaj4 & \textcolor{NavyBlue}{0.51} & \textcolor{Gray}{0.09} & \textcolor{Gray}{0.13} & \textcolor{Gray}{-0.01} & \textcolor{Gray}{-0.05} & 0.38  & 0.62  & 1.2 \\
    spok3 & \textcolor{Brown}{0.5} & \textcolor{Gray}{0.24} & \textcolor{Gray}{-0.02} & \textcolor{Gray}{0.08} & \textcolor{Gray}{0.12} & 0.55  & 0.45  & 1.6 \\
    spok4 & \textcolor{Brown}{0.46} & \textcolor{Gray}{0.1} & \textcolor{Gray}{0} & \textcolor{Gray}{0.16} & \textcolor{Gray}{0.19} & 0.49  & 0.51  & 1.7 \\
    crse3 & \textcolor{Orange}{0.39} & \textcolor{Gray}{0.02} & \textcolor{Gray}{0.12} & \textcolor{Gray}{0.11} & \textcolor{Gray}{-0.04} & 0.28  & 0.72  & 1.4 \\
    crse4 & \textcolor{Orange}{0.33} & \textcolor{Gray}{0.19} & \textcolor{Gray}{0.2} & \textcolor{Gray}{0.11} & \textcolor{Gray}{-0.21} & 0.39  & 0.61  & 3.5 \\
    crse2 & \textcolor{Orange}{0.32} & \textcolor{Gray}{0.26} & \textcolor{Gray}{0.02} & \textcolor{Gray}{-0.11} & \textcolor{Gray}{0.03} & 0.22  & 0.78  & 2.2 \\
    crse1 & \textcolor{Orange}{0.28} & \textcolor{Gray}{0.11} & \textcolor{Gray}{0.2} & \textcolor{Gray}{0.05} & \textcolor{Gray}{0.03} & 0.27  & 0.73  & 2.3 \\
    akce2 & \textcolor{Gray}{0} & \textcolor{NavyBlue}{0.77} & \textcolor{Gray}{-0.08} & \textcolor{Gray}{-0.01} & \textcolor{Gray}{0.02} & 0.56  & 0.44  & 1 \\
    akce1 & \textcolor{Gray}{0.04} & \textcolor{NavyBlue}{0.71} & \textcolor{Gray}{0.07} & \textcolor{Gray}{0.05} & \textcolor{Gray}{-0.11} & 0.6   & 0.4   & 1.1 \\
    akce4 & \textcolor{Gray}{0.04} & \textcolor{NavyBlue}{0.54} & \textcolor{Gray}{-0.03} & \textcolor{Gray}{0.07} & \textcolor{Gray}{0.18} & 0.41  & 0.59  & 1.3 \\
    akce3 & \textcolor{Gray}{-0.02} & \textcolor{NavyBlue}{0.51} & \textcolor{Gray}{-0.03} & \textcolor{Gray}{0.17} & \textcolor{Gray}{0.06} & 0.36  & 0.64  & 1.3 \\
    loaj1 & \textcolor{Gray}{0.29} & \textcolor{Brown}{0.46} & \textcolor{Gray}{0.06} & \textcolor{Gray}{0.07} & \textcolor{Gray}{-0.03} & 0.52  & 0.48  & 1.8 \\
    fipo4 & \textcolor{Gray}{0.05} & \textcolor{Brown}{0.43} & \textcolor{Gray}{0.38} & \textcolor{Gray}{-0.02} & \textcolor{Gray}{-0.08} & 0.44  & 0.56  & 2.1 \\
    spok2 & \textcolor{Gray}{0.16} & \textcolor{Orange}{0.42} & \textcolor{Gray}{-0.02} & \textcolor{Gray}{0.01} & \textcolor{Gray}{0.14} & 0.32  & 0.68  & 1.5 \\
    spok1 & \textcolor{Gray}{0.22} & \textcolor{Orange}{0.33} & \textcolor{Gray}{0.06} & \textcolor{Gray}{0.08} & \textcolor{Gray}{0.15} & 0.39  & 0.61  & 2.4 \\
    kval2 & \textcolor{Gray}{0.08} & \textcolor{Brown}{0.27} & \textcolor{Gray}{0.18} & \textcolor{Gray}{0.24} & \textcolor{Gray}{-0.16} & 0.33  & 0.67  & 3.7 \\
    inpr3 & \textcolor{Gray}{0.07} & \textcolor{Gray}{0.05} & \textcolor{NavyBlue}{0.7} & \textcolor{Gray}{-0.02} & \textcolor{Gray}{0.04} & 0.56  & 0.44  & 1 \\
    inpr1 & \textcolor{Gray}{-0.02} & \textcolor{Gray}{-0.14} & \textcolor{NavyBlue}{0.66} & \textcolor{Gray}{0.05} & \textcolor{Gray}{-0.11} & 0.39  & 0.61  & 1.2 \\
    fipo1 & \textcolor{Gray}{-0.04} & \textcolor{Gray}{0.2} & \textcolor{Orange}{0.54} & \textcolor{Gray}{0.03} & \textcolor{Gray}{0.24} & 0.54  & 0.46  & 1.7 \\
    inpr2 & \textcolor{Gray}{0.14} & \textcolor{Gray}{-0.1} & \textcolor{NavyBlue}{0.48} & \textcolor{Gray}{0.04} & \textcolor{Gray}{0.34} & 0.5   & 0.5   & 2.1 \\
    inpr4 & \textcolor{Gray}{0.03} & \textcolor{Gray}{-0.05} & \textcolor{NavyBlue}{0.47} & \textcolor{Gray}{0.17} & \textcolor{Gray}{-0.12} & 0.29  & 0.71  & 1.4 \\
    duve3 & \textcolor{Gray}{0} & \textcolor{Gray}{0} & \textcolor{Gray}{-0.09} & \textcolor{NavyBlue}{0.66} & \textcolor{Gray}{-0.01} & 0.39  & 0.61  & 1 \\
    duve2 & \textcolor{Gray}{0.03} & \textcolor{Gray}{-0.01} & \textcolor{Gray}{0} & \textcolor{NavyBlue}{0.44} & \textcolor{Gray}{0.08} & 0.24  & 0.76  & 1.1 \\
    kval3 & \textcolor{Gray}{0.11} & \textcolor{Gray}{-0.02} & \textcolor{Gray}{0.1} & \textcolor{Orange}{0.43} & \textcolor{Gray}{-0.11} & 0.27  & 0.73  & 1.4 \\
    kval1 & \textcolor{Gray}{0.05} & \textcolor{Gray}{0.24} & \textcolor{Gray}{0.18} & \textcolor{Orange}{0.38} & \textcolor{Gray}{0.11} & 0.51  & 0.49  & 2.4 \\
    duve1 & \textcolor{Gray}{-0.03} & \textcolor{Gray}{0.23} & \textcolor{Gray}{0.01} & \textcolor{NavyBlue}{0.35} & \textcolor{Gray}{-0.28} & 0.26  & 0.74  & 2.7 \\
    duve4 & \textcolor{Gray}{0.08} & \textcolor{Gray}{0.01} & \textcolor{Gray}{0.19} & \textcolor{NavyBlue}{0.35} & \textcolor{Gray}{0.23} & 0.38  & 0.62  & 2.5 \\
    kval4 & \textcolor{Gray}{0.18} & \textcolor{Gray}{0.01} & \textcolor{Gray}{0.16} & \textcolor{Orange}{0.3} & \textcolor{Gray}{0.25} & 0.42  & 0.58  & 3.3 \\
    fipo2 & \textcolor{Gray}{0.09} & \textcolor{Gray}{0.04} & \textcolor{Gray}{0.15} & \textcolor{Gray}{0.26} & \textcolor{NavyBlue}{0.41} & 0.44  & 0.56  & 2.1 \\
    fipo3 & \textcolor{Gray}{-0.01} & \textcolor{Gray}{0.25} & \textcolor{Gray}{0.28} & \textcolor{Gray}{0.07} & \textcolor{NavyBlue}{0.33} & 0.42  & 0.58  & 3 \\
    \end{tabular}%
  \label{tab:addlabel}%
\end{table}%





\section{Confirmatory Factor Analysis - structural model}
This model assesses relations between constructs and questions which determine them. It also allow first tests of fit. We distinguish between four fit indexes (FI):
\begin{enumerate}
\item Absolute FI \newline
these indexes describe proportion of total covariance explained by model. Higher values indicates better fit. However, KLINE states that high index value does not itself indicate the model is adequate (similar problem we face with $R^2$ in regression analysis.
\item Incremental FI \newline
are also called \emph{comparative fit indexes}. These indexes set the improvement of fit due to our theoretical model compared to null model. Null model states zero covariances between variables.
\item Parsimony-adjusted indexes \newline
The main purpose of these indexes is to overcome problems which was mentioned in point 1. These indexes penalise models with higher complexity by correction $fd$ (degrees of freedom). 
\item Predictive FI \newline
Prediction is closely related to reproducibility and replications on hypothetical data sample. \footnote{Although I can generate such a hypothetical sample using \texttt{psych} package (never did it, though!), it is not important for the majority of research papers, therefore can be excluded from our analysis.}
\end{enumerate}
\begin{verbatim}
lavaan (0.5-16) converged normally after  79 iterations

  Number of observations                           459

  Estimator                                       DWLS      Robust
  Minimum Function Test Statistic             1080.381    1151.167
  Degrees of freedom                               436         436
  P-value (Chi-square)                           0.000       0.000
  Scaling correction factor                                  1.170
  Shift parameter                                          227.384
    for simple second-order correction (Mplus variant)

Model test baseline model:

  Minimum Function Test Statistic            44361.224   12851.634
  Degrees of freedom                               496         496
  P-value                                        0.000       0.000

User model versus baseline model:

  Comparative Fit Index (CFI)                    0.985       0.942
  Tucker-Lewis Index (TLI)                       0.983       0.934

Root Mean Square Error of Approximation:

  RMSEA                                          0.057       0.060
  90 Percent Confidence Interval          0.053  0.061       0.056  0.064
  P-value RMSEA <= 0.05                          0.004       0.000

Weighted Root Mean Square Residual:

  WRMR                                           1.316       1.316
\end{verbatim}
The Bentler Comparative Fit Index (CFI) belongs to incremental fit index which compares this CFA model to the independence model. 
\begin{equation}
\text{CFI} = 1 - \frac{\chi^2_{\text{M}} - df_{\text{M}}}{\chi^2_{\text{Base}- df_{\text{Base}}}}
\end{equation}
KLINE states that CFI $\geq 0.95$ while SRMR (correlation residuals) $\leq 0.08$ can be considered as "acceptable" fit. Our level of CFI is 0.985 which indicates improvement of about 98.5\% to base (no-covariance) model. \smallskip

The Root Mean Square Error of Approximation (RMSEA) is scaled indicator. Values close 0 indicate better fit. This indicator is adjusted by complexity. RMSEA is calculated as:
\begin{equation}
\text{RMSEA}=\sqrt{\frac{\chi^2_{\text{M}} - df_{\text{M}}}{df_{\text{M}}\left(N-1\right)}}
\end{equation}
95\% confidence interval of RMSEA of our CFA model is $\left\langle 0.053, 0.061 \right\rangle$ for standard estimate and  $\left\langle 0.056, 0.066 \right\rangle$ for robust estimate. Statistical test is also provided.\footnote{TODO:KLINE, p.206)}

Path coefficients estimates can be shown either in plot or table. I present table visualisation as the model consists of eight constructs and therefore 28 edges between constructs should been drawn and labelled (CFA model assumes correlation relation between each pair of variables in model). 

% Table generated by Excel2LaTeX from sheet 'List1'
\begin{table}[htbp]
  \centering
  \caption{In the matrix path correlation are presented. All variables exhibit positive correlations. Some of theses relations are weak in strength (such an inpr and akce). Estimates of standard errors and following z-test show statistically significant results, though.}
    \begin{tabular}{l|lllllll}
          & akce  & crse  & duve  & fipo  & inpr  & kval  & loaj \\\hline
    crse  & 0.632 &       &       &       &       &       &  \\
    duve  & 0.678 & 0.637 &       &       &       &       &  \\
    fipo  & 0.713 & 0.701 & 0.747 &       &       &       &  \\
    inpr  & \textcolor{Red}{0.378} & 0.58  & 0.642 & 0.885 &       &       &  \\
    kval  & 0.75  & 0.768 & 0.922 & 0.885 & 0.729 &       &  \\
    loaj  & 0.71  & 0.844 & 0.653 & 0.647 & 0.537 & 0.763 &  \\
    spok  & 0.827 & 0.77  & 0.745 & 0.783 & 0.572 & 0.895 & 0.914 \\
    \end{tabular}%
  \label{t:cfa_paths}%
\end{table}%

\begin{lstlisting}
cfa.model <- "
akce =~ akce1 + akce2 + akce3 + akce4
crse =~ crse1 + crse2 + crse3 + crse4
duve =~ duve1 + duve2 + duve3 + duve4
fipo =~ fipo1 + fipo2 + fipo3 + fipo4
inpr =~ inpr1 + inpr2 + inpr3 + inpr4
kval =~ kval1 + kval2 + kval3 + kval4
loaj =~ loaj1 + loaj2 + loaj3 + loaj4
spok =~ spok1 + spok2 + spok3 + spok4"

cfa.fit <- cfa(cfa.model, data = data.sem, std.lv=TRUE) # fit a model
summary(cfa.fit, fit.measures=TRUE) # print results with statistical and performance tests
\end{lstlisting}



\subsection{Reliability of questions}
Reliability is an extent to which a variable is consistent in what is intended to measure. The most often used measure is Cronbach's alpha. Hair suggests that values higher 0.7 can be considered as reliable.\cite{hair}. \cite{stough} summarises  Nunnally's results published in 1978 as: \emph{$\alpha$ is recommended .70 only for early stage research. For basic research, Nunnally recommended a criterion of .80, and for clinical decision making a minimum reliability level of .90+ was encouraged.} Alternative to $\alpha$ is McDonald's omega ($\omega$) which is also provided here. Omega values are more flexible than $\alpha$ as it they do not rely on such a strict assumptions as $\alpha$ does. These are: tau-equivalence, absence of correlated error terms. \cite{stough} \newline 
Cronbach's alpha is computed as \footnote{In reliability(cfa.fit): The alpha is calculated from polychoric (polyserial) correlation not from Pearson's correlation.}:
\begin{equation}
\alpha=\frac{K}{K-1} \left(\frac{1-\sum^K_{k=1}\sigma^2_{Y_i}}{\sigma^2_x} \right)
\label{e:cron}
\end{equation}
where $\sum^K_{k=1}\sigma^2_{Y_i}$ is a sum of variances of inter-items which constitute composite score $X$. Variance of overall composite score is therefore $\sigma^2_x$.\newline
McDonald's omega $\omega_A$:
\begin{equation}
\omega_\text{A}=\frac{\left(\displaystyle\sum_{i=1}^k \lambda_i \right)^2}{\left(\displaystyle\sum_{i=1}^k \lambda_i \right)^2 + \displaystyle\sum_{i=1}^k \delta_{ii}}
\label{e:omegaa}
\end{equation}
value $\lambda_i$ is standardised factor loading and $\delta_{ii}$ is standardised error variance ($1-\delta_{ii}$).

\begin{lstlisting}
library(semTools)
reliability(cfa.fit)
\end{lstlisting}

% latex table generated in R 3.0.2 by xtable 1.7-3 package
% Mon Jul 07 15:40:49 2014
\begin{table}[ht]
\centering
\caption{Table presents several reliability measures. The firs \emph{alpha} refers to Cronbach's alpha. Set of question related to trustworthiness (\emph{duve}) can be considered as reliable.}
\begin{tabular}{rrrrrrrrr}
  \hline
 & akce & crse & duve & fipo & inpr & kval & loaj & spok \\ 
  \hline
alpha & 0.82 & 0.71 & 0.64 & 0.77 & 0.74 & 0.73 & 0.84 & 0.80 \\ 
  omega & 0.82 & 0.72 & 0.66 & 0.78 & 0.74 & 0.74 & 0.87 & 0.81 \\ 
  omega2 & 0.82 & 0.72 & 0.66 & 0.78 & 0.74 & 0.74 & 0.87 & 0.81 \\ 
  omega3 & 0.82 & 0.73 & 0.67 & 0.79 & 0.73 & 0.75 & 0.92 & 0.83 \\ 
   \hline
\end{tabular}
\label{t:reliab}
\end{table}

In the next step convergence among set of items representing underlying construct shall be checked. AVE stands for Average Variance Extracted and ranges from 0 to 1 (100\% of variance is explained by questions). Results from Cronbach should be supported by AVE. Recommended values are higher than 0.5. 

% Table generated by Excel2LaTeX from sheet 'List1'
\begin{table}[htbp]
  \centering
 \caption{AVE for individual question presents estimates of information conveyed in particular question with reference to other questions related to the same construct.}
    \begin{tabular}{r|r||r|rr}\hline
    akce1 & 0.674        & \textcolor{Red}{spok1} & \textcolor{Red}{0.482} \\
    akce2 & 0.538        & \textcolor{Red}{spok2} & \textcolor{Red}{0.347} \\
    \textcolor{Red}{akce3} & \textcolor{Red}{0.434}        & spok3 & 0.661 \\
    akce4 & 0.507        & spok4 & 0.615 \\\hline
    \textcolor{Red}{crse1} & \textcolor{Red}{0.398}        & fipo1 & 0.530 \\
    \textcolor{Red}{crse2} & \textcolor{Red}{0.304}        & \textcolor{Red}{fipo2} & \textcolor{Red}{0.444} \\
    \textcolor{Red}{crse3} & \textcolor{Red}{0.402}        & \textcolor{Red}{fipo3} & \textcolor{Red}{0.430} \\
    \textcolor{Red}{crse4} & \textcolor{Red}{0.486} &       \textcolor{Red}{fipo4} & \textcolor{Red}{0.477} \\\hline
    \textcolor{Red}{duve1}  & \textcolor{Red}{0.205}        & \textcolor{Red}{inpr1} & \textcolor{Red}{0.222} \\
    \textcolor{Red}{duve2} & \textcolor{Red}{0.281}        & inpr2 & 0.551 \\
    \textcolor{Red}{duve3} & \textcolor{Red}{0.368}        & inpr3 & 0.635 \\
    \textcolor{Red}{duve4} & \textcolor{Red}{0.465}        & \textcolor{Red}{inpr4} & \textcolor{Red}{0.316} \\\hline
    loaj1 & 0.666        & kval1 & 0.576 \\
    loaj2 & 0.695        & \textcolor{Red}{kval2} & \textcolor{Red}{0.390} \\
    loaj3 & 0.703        & \textcolor{Red}{kval3} & \textcolor{Red}{0.278} \\
    \textcolor{Red}{loaj4} & \textcolor{Red}{0.441}        & \textcolor{Red}{kval4} & \textcolor{Red}{0.437} \\\hline
    \end{tabular}%
  \label{t:ave_ind}%
\end{table}%

While Table \ref{t:ave_ind} highlights individual questions and their information content, reliability of overall construct can be provided by averaging individual AVE's. This is shown in Table \ref{t:ave_const}.

% latex table generated in R 3.1.1 by xtable 1.7-3 package
% Sun Jul 13 17:42:13 2014
\begin{table}[ht]
\centering
\caption{Ave of construct reveals low values when compared to recommended threshold of 0.5. Lower values indicated there is more noise than actual information.}

\begin{tabular}{rr}
  \hline
 const & average \\  \hline
 akce & 0.54 \\ 
 crse & 0.40 \\ 
 duve & 0.33 \\ 
 fipo & 0.47 \\ 
 inpr & 0.43 \\ 
 kval & 0.42 \\ 
 loaj & 0.63 \\ 
 spok & 0.53 \\ 
   \hline
\end{tabular}
\label{t:ave_const}
\end{table}

\subsection{Summated Scales Analysis}
After summated scales were created (TODO: check negative question!), correlation analysis can be made:
% latex table generated in R 3.1.1 by xtable 1.7-3 package
% Sun Jul 13 19:13:05 2014
\begin{table}[htb!]
\centering
\caption{Matrix of Pearson correlations. It's important to achieve high value which govern relations between constructs (not causal). Variances are presented on the diagonal. High variance suggest highest descriptive power in data-set. If our aim is to find dimensions which describe dataset the most, we would assign \emph{ak}, \emph{in} and \emph{lo}. } Variables as coded as follows:
\begin{tabular}{l	rrrrrrrr}
  \hline
 & ak & cr & du & fi & in & kv & lo & sp \\ \hline
  ak & 11.6921 &  &  &  &  &  &  &  \\ 
  cr & 0.44 & 8.4595 &  &  &  & &  &  \\ 
  du & 0.46 & 0.39 & 4.4747 &  &  &  &  &  \\ 
  fi & 0.54 & 0.49 & 0.47 & 8.0229 &  &  &  &  \\ 
  in & 0.25 & 0.37 & 0.37 & 0.58 & 10.6420 &  &  &  \\ 
  kv & 0.52 & 0.50 & 0.59 & 0.63 & 0.47 & 5.7946 &  &  \\ 
  lo & 0.56 & 0.65 & 0.46 & 0.51 & 0.40 & 0.58 & 10.8094 &  \\ 
  sp & 0.63 & 0.55 & 0.48 & 0.59 & 0.37 & 0.62 & 0.72 & 6.8957 \\ 
   \hline
\end{tabular}
\end{table}

Anti Image Matrix (AIM)is a matrix of partial correlations which. It is recommended to continue in analysis when values in AIM are higher than 0.7. But, according to psychopedia: \emph{Values that exceed 0.3 or so represent items that are correlated with each other above and beyond the factors. Hence, one of these items should be eliminated.}\cite{psychopedia} According to SAS glossary \emph{The anti-image correlation matrix ... is a matrix of the negatives of the partial correlations among variables}.\cite{sas_glossary} So we should interpret correlation between $r(ak, in)= 0.18$ (Akceptace cen klientem and Individualni pristup ke klientovi) as small negative relation between variable when influence of other variables is removed.
% latex table generated in R 3.1.1 by xtable 1.7-3 package
% Sun Jul 13 19:28:27 2014
\begin{table}[ht]
\centering
\caption{Anti Image Matrix does presents poor results of partial correlations. No pair-correlation reaches level of recommended 0.7 .}  

\begin{tabular}{lrrrrrrrr}
  \hline
 & ak & cr & du & fi & in & kv & lo & sp \\ 
  \hline
ak & 1.00 &  &  &  &  &  &  &  \\ 
  cr & -0.02 & 1.00 &  &  &  &  & &  \\ 
  du & -0.14 & -0.02 & 1.00 &  &  &  &  &  \\ 
  fi & -0.23 & -0.12 & -0.03 & 1.00 &  &  &  &  \\ 
  in & \textcolor{NavyBlue}{0.18} & -0.04 & -0.09 & \textcolor{Red}{-0.41} & 1.00 &  &  &  \\ 
  kv & -0.07 & -0.06 & -0.31 & -0.24 & -0.12 & 1.00 &  &  \\ 
  lo & -0.13 & -0.38 & -0.04 & \textcolor{NavyBlue}{0.08} & -0.12 & -0.10 & 1.00 &  \\ 
  sp & -0.26 & -0.05 & -0.04 & -0.17 & \textcolor{NavyBlue}{0.05} & -0.16 & \textcolor{Red}{-0.40} & 1.00 \\ 
   \hline
\end{tabular}
\end{table}

According to Kline if the observed relation between two variables is due to at least one other common cause (measured or unmeasured variable), their association is spurious. To address influence of other \emph{measured} variables on the relation of interest, partial correlation should be  computed. In equation (\ref{e:partial}) relation between $y$ and $x_1$ while adjusting for influence of $x_2$ is computed. 

\begin{equation}
r_{yx{_1}.x{_2}}=\frac{r_{yx_1}-r_{y_2}r_{x_1x_2}}{\sqrt{1-r_{x_1x_2}^2}}
\label{e:partial}
\end{equation}

\section{Structural Equation Model}
\subsection{First (simple) model}
\begin{lstlisting}
library(lavaan)

colnames(num.sem) <- c(substr(colnames(num.sem),5,12))
# Model I - initial model
sem.model <- "
akce =~ akce1 + akce2 + akce3 + akce4
crse =~ crse1 + crse2 + crse3 + crse4
duve =~ duve1 + duve2 + duve3 + duve4
fipo =~ fipo1 + fipo2 + fipo3 + fipo4
inpr =~ inpr1 + inpr2 + inpr3 + inpr4
kval =~ kval1 + kval2 + kval3 + kval4
loaj =~ loaj1 + loaj2 + loaj3 + loaj4
spok =~ spok1 + spok2 + spok3 + spok4

fipo ~ inpr
akce ~ inpr
duve ~ inpr
akce ~ fipo
spok ~ duve
spok ~ akce
spok ~ kval
kval ~ inpr
kval ~ duve
kval ~ fipo
loaj ~ spok
crse ~ spok
crse ~ loaj
"
\end{lstlisting}


\begin{figure}[ht]
\includegraphics[trim = 3.75cm 19cm 3.5cm 4.25cm, clip,width=1.00\textwidth]{PathPlot}
\caption{SEM model depicting regression between latent constructs. Positive values imply positive dependency in the way of arrow orientation. Some estimates were not significant (crossed text). Plot was typeset manually in \LaTeX. R function \texttt{semPlot::semPaths} looks promising but I failed to draw diagram which would clearly depict all relation and estimates.}
\label{f:first_model}
\end{figure}

\begin{verbatim}
lavaan (0.5-16) converged normally after 104 iterations

  Number of observations                           459

  Estimator                                         ML
  Minimum Function Test Statistic             1185.587
  Degrees of freedom                               451
  P-value (Chi-square)                           0.000

Model test baseline model:

  Minimum Function Test Statistic             5918.433
  Degrees of freedom                               496
  P-value                                        0.000

User model versus baseline model:

  Comparative Fit Index (CFI)                    0.865
  Tucker-Lewis Index (TLI)                       0.851

Loglikelihood and Information Criteria:

  Loglikelihood user model (H0)             -17872.531
  Loglikelihood unrestricted model (H1)     -17279.738

  Number of free parameters                         77
  Akaike (AIC)                               35899.062
  Bayesian (BIC)                             36216.999
  Sample-size adjusted Bayesian (BIC)        35972.624

Root Mean Square Error of Approximation:

  RMSEA                                          0.060
  90 Percent Confidence Interval          0.055  0.064
  P-value RMSEA <= 0.05                          0.000

Standardized Root Mean Square Residual:

  SRMR                                           0.062
\end{verbatim}

\subsection{Second model}

\begin{lstlisting}

# Model II - from 2014-7-27
sem.model <- "
akce =~ akce1 + akce2 + akce3 + akce4
crse =~ crse1 + crse2 + crse3 + crse4
duve =~ duve1 + duve2 + duve3 + duve4
fipo =~ fipo1 + fipo2 + fipo3 + fipo4
inpr =~ inpr1 + inpr2 + inpr3 + inpr4
kval =~ kval1 + kval2 + kval3 + kval4
loaj =~ loaj1 + loaj2 + loaj3 + loaj4
spok =~ spok1 + spok2 + spok3 + spok4

fipo ~ inpr # H1
duve ~ inpr # H4


duve ~ kval + spok + loaj # H7 + H10 + H14
akce ~ fipo + inpr + kval # H5 + H3 + H9
loaj ~ akce # H13
kval ~ fipo + inpr # H6 + H2
spok ~ akce + kval # H8 + H11
spok ~ loaj # H12
crse ~ kval + loaj + spok  # H15 + H17 + H16
"

fit.sem <- sem(sem.model, data = num.sem, std.lv=TRUE)
summary(fit.sem, standardized = TRUE,fit.measures=TRUE)
\end{lstlisting}

\begin{verbatim}
lavaan (0.5-16) converged normally after 116 iterations

  Number of observations                           459

  Estimator                                         ML
  Minimum Function Test Statistic             1138.220
  Degrees of freedom                               447
  P-value (Chi-square)                           0.000

Model test baseline model:

  Minimum Function Test Statistic             5918.433
  Degrees of freedom                               496
  P-value                                        0.000

User model versus baseline model:

  Comparative Fit Index (CFI)                    0.873
  Tucker-Lewis Index (TLI)                       0.859

Loglikelihood and Information Criteria:

  Loglikelihood user model (H0)             -17848.848
  Loglikelihood unrestricted model (H1)     -17279.738

  Number of free parameters                         81
  Akaike (AIC)                               35859.695
  Bayesian (BIC)                             36194.148
  Sample-size adjusted Bayesian (BIC)        35937.078

Root Mean Square Error of Approximation:

  RMSEA                                          0.058
  90 Percent Confidence Interval          0.054  0.062
  P-value RMSEA <= 0.05                          0.001

Standardized Root Mean Square Residual:

  SRMR                                           0.057
\end{verbatim}

From the printed output we can conclude that RMSEA is not ideally low (which would be $\sim$ 0.05), SRMR is not lower than 0.055 and CFI has decreased to 0.873. This is inconsistent with aforementioned rule of "appropriate" fit.


\begin{figure}[ht]
\includegraphics[trim = 1.75cm 19cm 5.5cm 4.25cm, clip,width=1.00\textwidth]{PathPlot2}
\caption{SEM model depicting regression between latent constructs. Statistically insignificant paths are written in \emph{italics} to make plot more readable. Interpretation is the same as in Figure \ref{f:first_model}.}
\end{figure}

\subsection{Model Comparison}
We can test whether additional complexity (new hypothesis - edges) in the second model brought significant benefits. 
\begin{lstlisting}
anova(fit.sem1, fit.sem2)
\end{lstlisting}
\begin{verbatim}
Chi Square Difference Test
          Df   AIC   BIC Chisq Chisq diff Df diff Pr(>Chisq)    
fit.sem2 447 35860 36194  1138                                  
fit.sem1 451 35899 36217  1186       47.4       4    1.3e-09 ***
\end{verbatim}

Although the complexity costs 4 degrees of freedom it brings improvement of 47.4 in terms of $\chi^2$ difference. This improvement is statistically significant as p-val $<0.01$. Therefore, the second model is preferred. 

\section{Next steps}
\begin{itemize}
\item We shall discuss membership of individual questions. The differences between satisfaction (spok) and loaylty (laoj) seems to be very narrow.
\item Some hypothesis do not exhibit expected directions. These edges are non-significant (if I were not indoctrinated by statistical theory, I would call it coincidence). The next step should be discussion about their importance in the model. Re-definition of constructs (which might be done by combining some questions from \emph{spok} and \emph{loaj} should be considered.
\item There are other topics not analysed here (as they exceeds my knowledge of studied phenomena). I would start with identification of moderating and mediation effects. 
\item Although a lot of work had been done before we reached this first conclusions, it's only one evolution step. These results can't be by any means considered as publication-ready! The purpose of these steps were to assess the quality of both measurement and structural model. Data need to be cleaned up at least by removing some non-reliable questions. 
\item Some codes do no work as expected. Maybe, we could consider using bootstrapped estimates, robust covariances, etc. This is more technical point but any suggestions are welcomed!
\item You can follow my progress at \url{https://github.com/luboRprojects}

\end{itemize}

\clearpage
\begin{thebibliography}{9}

\bibitem{hair}
  Hair, Joseph F. ,
  \emph{Multivariate Data Analysis. 7th ed.}.
  NJ: Pearson Prentice Hall Upper Saddle River
  7th edition,
  2010.

\bibitem{psychopedia}Item Deletion before Factor Analysis. \emph{Psychlopedia}. N.p., n.d. Web. 14 July 2014. \url{http://www.psych-it.com.au/Psychlopedia/article.asp?id=161}

\bibitem{sas_glossary} Stata: Data Analysis and Statistical Software. \emph{Stata Bookstore: Stata 12 Documentation}. N.p., n.d. Web. 14 July 2014 \url{http://www.stata.com/bookstore/stata12/pdf/mv_glossary.pdf}

\bibitem{stough} Stough, Con, Donald H. Saklofske, and James D. A. Parker. \emph{Assessing Emotional Intelligence: Theory, Research, and Applications}. Dordrecht: Springer, 2009. Print

\end{thebibliography}

\section*{Appendices}
% Table generated by Excel2LaTeX from sheet 'List1'
\begin{table}[htbp]
  \centering
    \begin{tabular}{ll|l}
    \hline\multicolumn{2}{c}{Abbreviations} & Construct \\\hline
    ak    & akce  & Akceptace cen klientem \\
    cr    & crse  & Cross-selling  \\
    du    & duve  & Duvera \\
    fi    & fipo  & Akceptace financnich potreb klienta \\
    in    & inpr  & Individualni pristup ke klientovi \\
    kv    & kval  & Kvalita \\
    lo    & loaj  & Loajalita \\
    sp    & spok  & Spokojenost \\
    \end{tabular}%
      \caption{Abbreviation of construct used in the document.}
  \label{tab:addlabel}%
\end{table}%

\section*{Misc}
Correlation matrix between construct loyalty and individual Cross-Selling questions.
\begin{table}[ht]
\centering
\begin{tabular}{rrrrrr}
  \hline
 & loyal & crse1 & crse2 & crse3 & crse4 \\ 
  \hline
loyal & 1.00 & $<0.01$ & $<0.01$ & $<0.01$ & $<0.01$ \\ 
  crse1 & 0.46 & 1 &  &  &  \\ 
  crse2 & 0.40 & 0.33 & 1.00 &  &  \\ 
  crse3 & 0.48 & 0.39 & 0.28 & 1.00 &  \\ 
  crse4 & 0.51 & 0.34 & 0.27 & 0.45 & 1.00 \\ 
   \hline
\end{tabular}
\caption{Correlation matrix accompanied with p-values. We found statistically significant correlations between pairs of variables.}
\end{table}

\end{document}

%\begin{minipage}{0.2\textwidth}
%\begin{tabular}{|c|c|c|}
%\hline
% A & B & C \\
%\hline
% 1 & 2 & 3  \\
%\hline 
% 4 & 5 & 6 \\
%\hline
%\end{tabular}
%\end{minipage}
%\begin{minipage}{0.2\textwidth}
%\begin{tabular}{c|c|c}
% A & B & C \\
%\hline
% 1 & 2 & 3  \\
%\hline 
% 4 & 5 & 6 \\
%\end{tabular}
%\end{minipage}
